
\documentclass[a4paper,12pt]{article}
\usepackage[brazil]{babel}
\usepackage[utf8]{inputenc}

\title{\textbf{Trabalho Prático de Compiladores I \\ Etapa 4 e final: Compilador Integrado \\Anexo I: Testes realizados e saídas}}
\author{
   Felipe Buzatti e Letícia Lana Cherchiglia \\
   \{buzatti,letslc\}@dcc.ufmg.br \\ \\
   \textit{Departamento de Ciência da Computação}\\
   \textit{Universidade Federal de Minas Gerais}\\
}
\date{20 de Junho de 2011}

\begin{document}
\maketitle

\section{Testes e saídas}
\subsection{teste1.txt}

\begin{footnotesize}\begin{verbatim}
program teste1
declare 
   integer a, b, c, d;
do
   read(b, c, d);
   a := b + -c * (b + d);
   write(a);
end
\end{verbatim}\end{footnotesize}

\subsection{Saída para valores de entrada 2,3,4}

\begin{footnotesize}\begin{verbatim}
Terminal: 
Entre com um numero: 2
Entre com um numero: 3
Entre com um numero: 4
-16

Arquivo:
INPP
AMEM 4
LEIT
ARMZ 0 1
LEIT
ARMZ 0 2
LEIT
ARMZ 0 3
CRVL 0 1
CRVL 0 2
INVR
CRVL 0 1
CRVL 0 3
SOMA
MULT
SOMA
ARMZ 0 0
CRVL 0 0
IMPR
PARA
\end{verbatim}\end{footnotesize}

\subsection{teste2.txt}

\begin{footnotesize}\begin{verbatim}
program teste2
declare
   integer p, q, a, b, c;
do
   p := 1;
   read(q, b);
   c := 2;
   if q < p 
   then
      a := b;
   else
      a := b * c;
   end;
   if a < 0
   then
      a := -a;
   end;
   write(a, b);
end
\end{verbatim}\end{footnotesize}

\subsection{Saída para valores de entrada 2,3}

\begin{footnotesize}\begin{verbatim}
Terminal: 
Entre com um numero: 2
Entre com um numero: 3
6
3

Arquivo:
INPP
AMEM 5
CRCT 1
ARMZ 0 0
LEIT
ARMZ 0 1
LEIT
ARMZ 0 3
CRCT 2
ARMZ 0 4
CRVL 0 1
CRVL 0 0
CMME
DSVF LIE0
CRVL 0 3
ARMZ 0 2
DSVS LIE1
LIE0 CRVL 0 3
CRVL 0 4
MULT
ARMZ 0 2
LIE1 CRVL 0 2
CRCT 0
CMME
DSVF LI0
CRVL 0 2
INVR
ARMZ 0 2
LI0 CRVL 0 2
IMPR
CRVL 0 3
IMPR
PARA
\end{verbatim}\end{footnotesize}

\subsection{teste3.txt}
\begin{footnotesize}\begin{verbatim}
program teste3
declare
   integer a, b, c;
   boolean x, y, z;
do
   read(c, b);
   z := b < c;
   y := true;
   a := 1;
   while c < b
   do
      a := a + b;
      b := b + -1;
      y := not y;
   end;
   z := c < a;
   if z 
   then 
      a := 0;
   end;
   write(a);
end
\end{verbatim}\end{footnotesize}

\subsection{Saída para valores de entrada 4,4}

\begin{footnotesize}\begin{verbatim}
Terminal: 
Entre com um numero: 4
Entre com um numero: 4
1

Arquivo:
INPP
AMEM 6
LEIT
ARMZ 0 2
LEIT
ARMZ 0 1
CRVL 0 1
CRVL 0 2
CMME
ARMZ 0 5
CRCT 1
ARMZ 0 4
CRCT 1
ARMZ 0 0
LW0 CRVL 0 2
CRVL 0 1
CMME
DSVF LW1
CRVL 0 0
CRVL 0 1
SOMA
ARMZ 0 0
CRVL 0 1
CRCT 1
INVR
SOMA
ARMZ 0 1
CRVL 0 4
NEGA
ARMZ 0 4
DSVS LW0
LW1 CRVL 0 2
CRVL 0 0
CMME
ARMZ 0 5
CRVL 0 5
DSVF LI0
CRCT 0
ARMZ 0 0
LI0 CRVL 0 0
IMPR
PARA
\end{verbatim}\end{footnotesize}

\subsection{testeA.txt}
\begin{footnotesize}\begin{verbatim}
program testeA
declare
   integer a, b;
   boolean c;
do 
   read(a, b);
   c:= a < b*3;
   write(c);
end
\end{verbatim}\end{footnotesize}

\subsection{Saída para valores de entrada 10,2}

\begin{footnotesize}\begin{verbatim}
Terminal: 
Entre com um numero: 10
Entre com um numero: 2
0

Arquivo:
INPP
AMEM 3
LEIT
ARMZ 0 0
LEIT
ARMZ 0 1
CRVL 0 0
CRVL 0 1
CRCT 3
MULT
CMME
ARMZ 0 2
CRVL 0 2
IMPR
PARA
\end{verbatim}\end{footnotesize}

\subsection{testeB.txt}

\begin{footnotesize}\begin{verbatim}
program testeB
declare
   integer i, j, k,l;
   boolean b;
do 
   i := 4 * (5+3) * 5;
   j := 100 + i;
   k := i * j;
   l := i * j + k;
   b := k = l;  
   write(k,l,b);
end
\end{verbatim}\end{footnotesize}

\subsection{Saída}

\begin{footnotesize}\begin{verbatim}
Terminal: 
41600
83200
0

Arquivo:
INPP
AMEM 5
CRCT 4
CRCT 5
CRCT 3
SOMA
MULT
CRCT 5
MULT
ARMZ 0 0
CRCT 100
CRVL 0 0
SOMA
ARMZ 0 1
CRVL 0 0
CRVL 0 1
MULT
ARMZ 0 2
CRVL 0 0
CRVL 0 1
MULT
CRVL 0 2
SOMA
ARMZ 0 3
CRVL 0 2
CRVL 0 3
CMIG
ARMZ 0 4
CRVL 0 2
IMPR
CRVL 0 3
IMPR
CRVL 0 4
IMPR
PARA
\end{verbatim}\end{footnotesize}

\subsection{testeA.txt}

\begin{footnotesize}\begin{verbatim}
\end{verbatim}\end{footnotesize}

\subsection{Saída para valores de entrada 10,2}

\begin{footnotesize}\begin{verbatim}
\end{verbatim}\end{footnotesize}


\end{document}

