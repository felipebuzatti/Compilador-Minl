
\documentclass[a4paper,12pt]{article}
\usepackage[brazil]{babel}
\usepackage[utf8]{inputenc}

\title{\textbf{Trabalho Prático de Compiladores I \\ Etapa 4 e final: Compilador Integrado \\Anexo I: Testes realizados e saídas}}
\author{
   Felipe Buzatti e Letícia Lana Cherchiglia \\
   \{buzatti,letslc\}@dcc.ufmg.br \\ \\
   \textit{Departamento de Ciência da Computação}\\
   \textit{Universidade Federal de Minas Gerais}\\
}
\date{29 de Junho de 2011}

\begin{document}
\maketitle

\section{Testes e saídas}
\subsection{exemplo.txt}
\begin{footnotesize}\begin{verbatim}
program exemplo
declare 
   integer a,b,c;
do
   a := 1;
   b := 2;
   c := (-b + 1)* (b+a);
   write(c);
end
\end{verbatim}\end{footnotesize}

\subsection{Saída}
Terminal: 
\begin{footnotesize}\begin{verbatim}
-3
\end{verbatim}\end{footnotesize}

Arquivo:
\begin{footnotesize}\begin{verbatim}
INPP
AMEM 3
CRCT 1
ARMZ 0 0
CRCT 2
ARMZ 0 1
CRVL 0 1
INVR
CRCT 1
SOMA
CRVL 0 1
CRVL 0 0
SOMA
MULT
ARMZ 0 2
CRVL 0 2
IMPR
PARA
\end{verbatim}\end{footnotesize}

\subsection{teste1.txt}

\begin{footnotesize}\begin{verbatim}
program teste1
declare 
   integer a,b;
   real Felipe;
   boolean c;
   char d;
do
   a := 1;
   b := 2;
   d := 'e';
   Felipe := 1.1;
   write(a,b, d, Felipe);
end
\end{verbatim}\end{footnotesize}

\subsection{Saída}

Terminal: 
\begin{footnotesize}\begin{verbatim}
1
2
e
1.06619e+09
\end{verbatim}\end{footnotesize}

Arquivo:
\begin{footnotesize}\begin{verbatim}
INPP
AMEM 5
CRCT 1
ARMZ 0 0
CRCT 2
ARMZ 0 1
CRCT 101
ARMZ 0 4
CRCF 1.100000
ARMZ 0 2
CRVL 0 0
IMPR
CRVL 0 1
IMPR
CRVL 0 4
IMPC
CRVL 0 2
IMPF
PARA
\end{verbatim}\end{footnotesize}

\subsection{teste2.txt}

\begin{footnotesize}\begin{verbatim}
program teste2
declare
   integer p, q, a, b, c;
do
   p := 1;
   read(q, b);
   c := 2;
   if q < p 
   then
      a := b;
   else
      a := b * c;
   end;
   if a < 0
   then
      a := -a;
   end;
   write(a, b);
end
\end{verbatim}\end{footnotesize}

\subsection{Saída para valores de entrada 2,3}
Terminal: 

\begin{footnotesize}\begin{verbatim}
Entre com um numero: 2
Entre com um numero: 3
6
3
\end{verbatim}\end{footnotesize}

Arquivo:
\begin{footnotesize}\begin{verbatim}
INPP
AMEM 5
CRCT 1
ARMZ 0 0
LEIT
ARMZ 0 1
LEIT
ARMZ 0 3
CRCT 2
ARMZ 0 4
CRVL 0 1
CRVL 0 0
CMME
DSVF LIE0
CRVL 0 3
ARMZ 0 2
DSVS LIE1
LIE0 CRVL 0 3
CRVL 0 4
MULT
ARMZ 0 2
LIE1 CRVL 0 2
CRCT 0
CMME
DSVF LI0
CRVL 0 2
INVR
ARMZ 0 2
LI0 CRVL 0 2
IMPR
CRVL 0 3
IMPR
PARA
\end{verbatim}\end{footnotesize}

\subsection{teste3.txt}
\begin{footnotesize}\begin{verbatim}
program teste3
declare
   integer a, b, c;
   integer resultado;
   boolean x, y, z;
do
   read(c, b);
   y := true;
   a := 1;
   resultado := a + b+c*5;
   while c < b
   do
      a := a + b;
      b := b + -1;
      y := not y;
   end;
   if z 
   then 
      a := 0;
   end;
   write(a);
end
\end{verbatim}\end{footnotesize}

\subsection{Saída para valores de entrada 4,4}

Terminal: 
\begin{footnotesize}\begin{verbatim}
Entre com um numero: 4
Entre com um numero: 4
1
\end{verbatim}\end{footnotesize}

Arquivo:
\begin{footnotesize}\begin{verbatim}
INPP
AMEM 7
LEIT
ARMZ 0 2
LEIT
ARMZ 0 1
CRCT 1
ARMZ 0 5
CRCT 1
ARMZ 0 0
CRVL 0 0
CRVL 0 1
SOMA
CRVL 0 2
CRCT 5
MULT
SOMA
ARMZ 0 3
LW0 CRVL 0 2
CRVL 0 1
CMME
DSVF LW1
CRVL 0 0
CRVL 0 1
SOMA
ARMZ 0 0
CRVL 0 1
CRCT 1
INVR
SOMA
ARMZ 0 1
CRVL 0 5
NEGA
ARMZ 0 5
DSVS LW0
LW1 CRVL 0 6
DSVF LI0
CRCT 0
ARMZ 0 0
LI0 CRVL 0 0
IMPR
PARA
\end{verbatim}\end{footnotesize}

\subsection{teste4.txt}

\begin{footnotesize}\begin{verbatim}
program teste4
declare
   integer a, b, c;
   procedure p(integer x; integer y; integer z;)
   do
      z := x + y + z;
      write(x, z);
   end;
do
   a := 5;
   b := 8;
   c := 3;
   p(a, b, c);
   p(7, a+b+c, a);
end
\end{verbatim}\end{footnotesize}

\subsection{Saída}
Terminal: 
\begin{footnotesize}\begin{verbatim}
5
16
7
28
\end{verbatim}\end{footnotesize}

Arquivo:
\begin{footnotesize}\begin{verbatim}
INPP
AMEM 3
DSVS LFP0
LIP0 ENTR 1
CRVL 1 -7
CRVL 1 -6
SOMA
CRVL 1 -5
SOMA
ARMZ 1 -5
CRVL 1 -7
IMPR
CRVL 1 -5
IMPR
RTPR 3
LFP0 CRCT 5
ARMZ 0 0
CRCT 8
ARMZ 0 1
CRCT 3
ARMZ 0 2
CRVL 0 0
CRVL 0 1
CRVL 0 2
CHPR LIP0 0
CRCT 7
CRVL 0 0
CRVL 0 1
SOMA
CRVL 0 2
SOMA
CRVL 0 0
CHPR LIP0 0
PARA
\end{verbatim}\end{footnotesize}

\subsection{teste5.txt}
\begin{footnotesize}\begin{verbatim}
program teste5
declare
   integer i,j,k,l;
   boolean b;
do 
   i := 4 * (5+3) * 5;
   j := 100 + i;
   k := i * j;
   l := i * j + k;
   b := k = l;
   write(k,l,b);
end
\end{verbatim}\end{footnotesize}

\subsection{Saída}
Terminal: 
\begin{footnotesize}\begin{verbatim}
41600
83200
0
\end{verbatim}\end{footnotesize}

Arquivo:
\begin{footnotesize}\begin{verbatim}
INPP
AMEM 5
CRCT 4
CRCT 5
CRCT 3
SOMA
MULT
CRCT 5
MULT
ARMZ 0 0
CRCT 100
CRVL 0 0
SOMA
ARMZ 0 1
CRVL 0 0
CRVL 0 1
MULT
ARMZ 0 2
CRVL 0 0
CRVL 0 1
MULT
CRVL 0 2
SOMA
ARMZ 0 3
CRVL 0 2
CRVL 0 3
CMIG
ARMZ 0 4
CRVL 0 2
IMPR
CRVL 0 3
IMPR
CRVL 0 4
IMPR
PARA
\end{verbatim}\end{footnotesize}

\subsection{teste6.txt}

\begin{footnotesize}\begin{verbatim}
program teste6
declare
   integer divisor, number;
   boolean nofactor, prime, divisor;
do 
   read(number);
   write(number);
   divisor := number;
   nofactor := true;
   if (prime + 1)
   then
      divisor := divisor - 1;
   end;
end
\end{verbatim}\end{footnotesize}

\subsection{Saída}
Terminal: 
\begin{footnotesize}\begin{verbatim}
ERRO! Redefinição de tipo da variável 'divisor'! Abortando...

./MEPA.elf saida.txt 0 0
MEPA.elf: program.cpp:257: int Program::next(): Assertion `i >= 0 && i < n' failed.
make: *** [run] Abortado
\end{verbatim}\end{footnotesize}

Arquivo: não foi gerado, pois houve um erro, a redefinição da variável 'divisor'.

\subsection{teste7.txt}

\begin{footnotesize}\begin{verbatim}
program teste7
declare
   integer divisor, number;
   boolean nofactor, prime;
do 
   read(number);
   write(number);
   divisor := number;
   nofactor := true;
   if (prime + 1)
   then
      divisor := divisor - 1;
   end;
end
\end{verbatim}\end{footnotesize}

\subsection{Saída}
Terminal: 
\begin{footnotesize}\begin{verbatim}
ERRO! Soma incompatível (tipos diferentes)! Abortando...

./MEPA.elf saida.txt 0 0
MEPA.elf: program.cpp:257: int Program::next(): Assertion `i >= 0 && i < n' failed.
make: *** [run] Abortado
\end{verbatim}\end{footnotesize}

Arquivo: não foi gerado, pois houve um erro, a soma de um boolean ('prime') com um valor inteiro (1) no 'if'.

\subsection{teste8.txt}

\begin{footnotesize}\begin{verbatim}
program teste8
declare 
   integer a,b;
   real d;
do
   a := 1;
   b := 2;
   d := a+b;
end
\end{verbatim}\end{footnotesize}

\subsection{Saída}
Terminal: 
\begin{footnotesize}\begin{verbatim}
ERRO! Atribuição inválida na variável 'd'! Abortando...

./MEPA.elf saida.txt 0 0
MEPA.elf: program.cpp:257: int Program::next(): Assertion `i >= 0 && i < n' failed.
make: *** [run] Abortado
\end{verbatim}\end{footnotesize}

Arquivo: não foi gerado, pois houve um erro, a atribuição de valor inteiro a uma variável declarada como real.

\subsection{teste9.txt}

\begin{footnotesize}\begin{verbatim}
program teste9
declare
	integer a, b, c;
	procedure p(integer x; integer y; integer z;)
	do
		z := x + y + z;
		write(x, z);
	end;
do
	a := 5;
	b := 8;
	c := 3;
	p(a, b, c);
	p(7, a+b+c);
end
\end{verbatim}\end{footnotesize}

\subsection{Saída}
Terminal: 
\begin{footnotesize}\begin{verbatim}
ERRO! Número de parâmetros na chamada do processo 'p' incompatível. ! Abortando...

./MEPA.elf saida.txt 0 0
MEPA.elf: program.cpp:257: int Program::next(): Assertion `i >= 0 && i < n' failed.
make: *** [run] Abortado
\end{verbatim}\end{footnotesize}

Arquivo: não foi gerado, pois houve um erro, a passagem de apenas 2 parâmetros ao invés de 3.

\end{document}

